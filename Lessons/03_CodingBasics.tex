\PassOptionsToPackage{unicode=true}{hyperref} % options for packages loaded elsewhere
\PassOptionsToPackage{hyphens}{url}
%
\documentclass[]{article}
\usepackage{lmodern}
\usepackage{amssymb,amsmath}
\usepackage{ifxetex,ifluatex}
\usepackage{fixltx2e} % provides \textsubscript
\ifnum 0\ifxetex 1\fi\ifluatex 1\fi=0 % if pdftex
  \usepackage[T1]{fontenc}
  \usepackage[utf8]{inputenc}
  \usepackage{textcomp} % provides euro and other symbols
\else % if luatex or xelatex
  \usepackage{unicode-math}
  \defaultfontfeatures{Ligatures=TeX,Scale=MatchLowercase}
\fi
% use upquote if available, for straight quotes in verbatim environments
\IfFileExists{upquote.sty}{\usepackage{upquote}}{}
% use microtype if available
\IfFileExists{microtype.sty}{%
\usepackage[]{microtype}
\UseMicrotypeSet[protrusion]{basicmath} % disable protrusion for tt fonts
}{}
\IfFileExists{parskip.sty}{%
\usepackage{parskip}
}{% else
\setlength{\parindent}{0pt}
\setlength{\parskip}{6pt plus 2pt minus 1pt}
}
\usepackage{hyperref}
\hypersetup{
            pdftitle={3: Coding Basics},
            pdfauthor={Environmental Data Analytics \textbar{} Kateri Salk},
            pdfborder={0 0 0},
            breaklinks=true}
\urlstyle{same}  % don't use monospace font for urls
\usepackage[margin=2.54cm]{geometry}
\usepackage{color}
\usepackage{fancyvrb}
\newcommand{\VerbBar}{|}
\newcommand{\VERB}{\Verb[commandchars=\\\{\}]}
\DefineVerbatimEnvironment{Highlighting}{Verbatim}{commandchars=\\\{\}}
% Add ',fontsize=\small' for more characters per line
\usepackage{framed}
\definecolor{shadecolor}{RGB}{248,248,248}
\newenvironment{Shaded}{\begin{snugshade}}{\end{snugshade}}
\newcommand{\AlertTok}[1]{\textcolor[rgb]{0.94,0.16,0.16}{#1}}
\newcommand{\AnnotationTok}[1]{\textcolor[rgb]{0.56,0.35,0.01}{\textbf{\textit{#1}}}}
\newcommand{\AttributeTok}[1]{\textcolor[rgb]{0.77,0.63,0.00}{#1}}
\newcommand{\BaseNTok}[1]{\textcolor[rgb]{0.00,0.00,0.81}{#1}}
\newcommand{\BuiltInTok}[1]{#1}
\newcommand{\CharTok}[1]{\textcolor[rgb]{0.31,0.60,0.02}{#1}}
\newcommand{\CommentTok}[1]{\textcolor[rgb]{0.56,0.35,0.01}{\textit{#1}}}
\newcommand{\CommentVarTok}[1]{\textcolor[rgb]{0.56,0.35,0.01}{\textbf{\textit{#1}}}}
\newcommand{\ConstantTok}[1]{\textcolor[rgb]{0.00,0.00,0.00}{#1}}
\newcommand{\ControlFlowTok}[1]{\textcolor[rgb]{0.13,0.29,0.53}{\textbf{#1}}}
\newcommand{\DataTypeTok}[1]{\textcolor[rgb]{0.13,0.29,0.53}{#1}}
\newcommand{\DecValTok}[1]{\textcolor[rgb]{0.00,0.00,0.81}{#1}}
\newcommand{\DocumentationTok}[1]{\textcolor[rgb]{0.56,0.35,0.01}{\textbf{\textit{#1}}}}
\newcommand{\ErrorTok}[1]{\textcolor[rgb]{0.64,0.00,0.00}{\textbf{#1}}}
\newcommand{\ExtensionTok}[1]{#1}
\newcommand{\FloatTok}[1]{\textcolor[rgb]{0.00,0.00,0.81}{#1}}
\newcommand{\FunctionTok}[1]{\textcolor[rgb]{0.00,0.00,0.00}{#1}}
\newcommand{\ImportTok}[1]{#1}
\newcommand{\InformationTok}[1]{\textcolor[rgb]{0.56,0.35,0.01}{\textbf{\textit{#1}}}}
\newcommand{\KeywordTok}[1]{\textcolor[rgb]{0.13,0.29,0.53}{\textbf{#1}}}
\newcommand{\NormalTok}[1]{#1}
\newcommand{\OperatorTok}[1]{\textcolor[rgb]{0.81,0.36,0.00}{\textbf{#1}}}
\newcommand{\OtherTok}[1]{\textcolor[rgb]{0.56,0.35,0.01}{#1}}
\newcommand{\PreprocessorTok}[1]{\textcolor[rgb]{0.56,0.35,0.01}{\textit{#1}}}
\newcommand{\RegionMarkerTok}[1]{#1}
\newcommand{\SpecialCharTok}[1]{\textcolor[rgb]{0.00,0.00,0.00}{#1}}
\newcommand{\SpecialStringTok}[1]{\textcolor[rgb]{0.31,0.60,0.02}{#1}}
\newcommand{\StringTok}[1]{\textcolor[rgb]{0.31,0.60,0.02}{#1}}
\newcommand{\VariableTok}[1]{\textcolor[rgb]{0.00,0.00,0.00}{#1}}
\newcommand{\VerbatimStringTok}[1]{\textcolor[rgb]{0.31,0.60,0.02}{#1}}
\newcommand{\WarningTok}[1]{\textcolor[rgb]{0.56,0.35,0.01}{\textbf{\textit{#1}}}}
\usepackage{graphicx,grffile}
\makeatletter
\def\maxwidth{\ifdim\Gin@nat@width>\linewidth\linewidth\else\Gin@nat@width\fi}
\def\maxheight{\ifdim\Gin@nat@height>\textheight\textheight\else\Gin@nat@height\fi}
\makeatother
% Scale images if necessary, so that they will not overflow the page
% margins by default, and it is still possible to overwrite the defaults
% using explicit options in \includegraphics[width, height, ...]{}
\setkeys{Gin}{width=\maxwidth,height=\maxheight,keepaspectratio}
\setlength{\emergencystretch}{3em}  % prevent overfull lines
\providecommand{\tightlist}{%
  \setlength{\itemsep}{0pt}\setlength{\parskip}{0pt}}
\setcounter{secnumdepth}{0}
% Redefines (sub)paragraphs to behave more like sections
\ifx\paragraph\undefined\else
\let\oldparagraph\paragraph
\renewcommand{\paragraph}[1]{\oldparagraph{#1}\mbox{}}
\fi
\ifx\subparagraph\undefined\else
\let\oldsubparagraph\subparagraph
\renewcommand{\subparagraph}[1]{\oldsubparagraph{#1}\mbox{}}
\fi

% set default figure placement to htbp
\makeatletter
\def\fps@figure{htbp}
\makeatother


\title{3: Coding Basics}
\author{Environmental Data Analytics \textbar{} Kateri Salk}
\date{Spring 2020}

\begin{document}
\maketitle

\hypertarget{objectives}{%
\subsection{Objectives}\label{objectives}}

\begin{enumerate}
\def\labelenumi{\arabic{enumi}.}
\tightlist
\item
  Discuss and navigate different data types in R
\item
  Create, manipulate, and explore datasets
\item
  Call packages in R
\end{enumerate}

\hypertarget{data-types-in-r}{%
\subsection{Data Types in R}\label{data-types-in-r}}

R treats objects differently based on their characteristics. For more
information, please see:
\url{https://www.statmethods.net/input/datatypes.html}.

\begin{itemize}
\item
  \textbf{Vectors} 1 dimensional structure that contains elements of the
  same type.
\item
  \textbf{Matrices} 2 dimensional structure that contains elements of
  the same type.
\item
  \textbf{Arrays} Similar to matrices, but can have more than 2
  dimensions. We will not delve into arrays in depth.
\item
  \textbf{Lists} Ordered collection of elements that can have different
  modes.
\item
  \textbf{Data Frames} 2 dimensional structure that is more general than
  a matrix. Columns can have different modes (e.g., numeric and factor).
  When we import csv files into the R workspace, they will enter as data
  frames.
\end{itemize}

Define what each new piece of syntax does below (i.e., fill in blank
comments). Note that the R chunk has been divided into sections (\# at
beginning of line, ---- at end)

\begin{Shaded}
\begin{Highlighting}[]
\CommentTok{# Vectors ----}
\NormalTok{vector1 <-}\StringTok{ }\KeywordTok{c}\NormalTok{(}\DecValTok{1}\NormalTok{,}\DecValTok{2}\NormalTok{,}\FloatTok{5.3}\NormalTok{,}\DecValTok{6}\NormalTok{,}\OperatorTok{-}\DecValTok{2}\NormalTok{,}\DecValTok{4}\NormalTok{) }\CommentTok{# numeric vector}
\NormalTok{vector1}
\end{Highlighting}
\end{Shaded}

\begin{verbatim}
## [1]  1.0  2.0  5.3  6.0 -2.0  4.0
\end{verbatim}

\begin{Shaded}
\begin{Highlighting}[]
\NormalTok{vector2 <-}\StringTok{ }\KeywordTok{c}\NormalTok{(}\StringTok{"one"}\NormalTok{,}\StringTok{"two"}\NormalTok{,}\StringTok{"three"}\NormalTok{) }\CommentTok{# character vector}
\NormalTok{vector2}
\end{Highlighting}
\end{Shaded}

\begin{verbatim}
## [1] "one"   "two"   "three"
\end{verbatim}

\begin{Shaded}
\begin{Highlighting}[]
\NormalTok{vector3 <-}\StringTok{ }\KeywordTok{c}\NormalTok{(}\OtherTok{TRUE}\NormalTok{,}\OtherTok{TRUE}\NormalTok{,}\OtherTok{TRUE}\NormalTok{,}\OtherTok{FALSE}\NormalTok{,}\OtherTok{TRUE}\NormalTok{,}\OtherTok{FALSE}\NormalTok{) }\CommentTok{#logical vector}
\NormalTok{vector3}
\end{Highlighting}
\end{Shaded}

\begin{verbatim}
## [1]  TRUE  TRUE  TRUE FALSE  TRUE FALSE
\end{verbatim}

\begin{Shaded}
\begin{Highlighting}[]
\NormalTok{vector1[}\DecValTok{3}\NormalTok{] }\CommentTok{# }
\end{Highlighting}
\end{Shaded}

\begin{verbatim}
## [1] 5.3
\end{verbatim}

\begin{Shaded}
\begin{Highlighting}[]
\CommentTok{# Matrices ----}
\NormalTok{matrix1 <-}\StringTok{ }\KeywordTok{matrix}\NormalTok{(}\DecValTok{1}\OperatorTok{:}\DecValTok{20}\NormalTok{, }\DataTypeTok{nrow =} \DecValTok{5}\NormalTok{,}\DataTypeTok{ncol =} \DecValTok{4}\NormalTok{) }\CommentTok{# }
\NormalTok{matrix1}
\end{Highlighting}
\end{Shaded}

\begin{verbatim}
##      [,1] [,2] [,3] [,4]
## [1,]    1    6   11   16
## [2,]    2    7   12   17
## [3,]    3    8   13   18
## [4,]    4    9   14   19
## [5,]    5   10   15   20
\end{verbatim}

\begin{Shaded}
\begin{Highlighting}[]
\NormalTok{matrix2 <-}\StringTok{ }\KeywordTok{matrix}\NormalTok{(}\DecValTok{1}\OperatorTok{:}\DecValTok{20}\NormalTok{, }\DataTypeTok{nrow =} \DecValTok{5}\NormalTok{, }\DataTypeTok{ncol =} \DecValTok{4}\NormalTok{, }\DataTypeTok{byrow =} \OtherTok{TRUE}\NormalTok{) }\CommentTok{#}
\NormalTok{matrix2}
\end{Highlighting}
\end{Shaded}

\begin{verbatim}
##      [,1] [,2] [,3] [,4]
## [1,]    1    2    3    4
## [2,]    5    6    7    8
## [3,]    9   10   11   12
## [4,]   13   14   15   16
## [5,]   17   18   19   20
\end{verbatim}

\begin{Shaded}
\begin{Highlighting}[]
\NormalTok{matrix3 <-}\StringTok{ }\KeywordTok{matrix}\NormalTok{(}\DecValTok{1}\OperatorTok{:}\DecValTok{20}\NormalTok{, }\DataTypeTok{nrow =} \DecValTok{5}\NormalTok{, }\DataTypeTok{ncol =} \DecValTok{4}\NormalTok{, }\DataTypeTok{byrow =} \OtherTok{TRUE}\NormalTok{, }\CommentTok{# return after comma continues the line}
                  \DataTypeTok{dimnames =} \KeywordTok{list}\NormalTok{(}\KeywordTok{c}\NormalTok{(}\StringTok{"uno"}\NormalTok{, }\StringTok{"dos"}\NormalTok{, }\StringTok{"tres"}\NormalTok{, }\StringTok{"cuatro"}\NormalTok{, }\StringTok{"cinco"}\NormalTok{), }
                                  \KeywordTok{c}\NormalTok{(}\StringTok{"un"}\NormalTok{, }\StringTok{"deux"}\NormalTok{, }\StringTok{"trois"}\NormalTok{, }\StringTok{"cat"}\NormalTok{))) }\CommentTok{#}

\NormalTok{matrix1[}\DecValTok{4}\NormalTok{, ] }\CommentTok{#}
\end{Highlighting}
\end{Shaded}

\begin{verbatim}
## [1]  4  9 14 19
\end{verbatim}

\begin{Shaded}
\begin{Highlighting}[]
\NormalTok{matrix1[ , }\DecValTok{3}\NormalTok{] }\CommentTok{#}
\end{Highlighting}
\end{Shaded}

\begin{verbatim}
## [1] 11 12 13 14 15
\end{verbatim}

\begin{Shaded}
\begin{Highlighting}[]
\NormalTok{matrix1[}\KeywordTok{c}\NormalTok{(}\DecValTok{12}\NormalTok{, }\DecValTok{14}\NormalTok{)] }\CommentTok{#}
\end{Highlighting}
\end{Shaded}

\begin{verbatim}
## [1] 12 14
\end{verbatim}

\begin{Shaded}
\begin{Highlighting}[]
\NormalTok{matrix1[}\KeywordTok{c}\NormalTok{(}\DecValTok{12}\OperatorTok{:}\DecValTok{14}\NormalTok{)] }\CommentTok{#}
\end{Highlighting}
\end{Shaded}

\begin{verbatim}
## [1] 12 13 14
\end{verbatim}

\begin{Shaded}
\begin{Highlighting}[]
\NormalTok{matrix1[}\DecValTok{2}\OperatorTok{:}\DecValTok{4}\NormalTok{, }\DecValTok{1}\OperatorTok{:}\DecValTok{3}\NormalTok{] }\CommentTok{#}
\end{Highlighting}
\end{Shaded}

\begin{verbatim}
##      [,1] [,2] [,3]
## [1,]    2    7   12
## [2,]    3    8   13
## [3,]    4    9   14
\end{verbatim}

\begin{Shaded}
\begin{Highlighting}[]
\NormalTok{cells <-}\StringTok{ }\KeywordTok{c}\NormalTok{(}\DecValTok{1}\NormalTok{, }\DecValTok{26}\NormalTok{, }\DecValTok{24}\NormalTok{, }\DecValTok{68}\NormalTok{)}
\NormalTok{rnames <-}\StringTok{ }\KeywordTok{c}\NormalTok{(}\StringTok{"R1"}\NormalTok{, }\StringTok{"R2"}\NormalTok{)}
\NormalTok{cnames <-}\StringTok{ }\KeywordTok{c}\NormalTok{(}\StringTok{"C1"}\NormalTok{, }\StringTok{"C2"}\NormalTok{) }
\NormalTok{matrix4 <-}\StringTok{ }\KeywordTok{matrix}\NormalTok{(cells, }\DataTypeTok{nrow =} \DecValTok{2}\NormalTok{, }\DataTypeTok{ncol =} \DecValTok{2}\NormalTok{, }\DataTypeTok{byrow =} \OtherTok{TRUE}\NormalTok{,}
  \DataTypeTok{dimnames =} \KeywordTok{list}\NormalTok{(rnames, cnames)) }\CommentTok{# }
\NormalTok{matrix4}
\end{Highlighting}
\end{Shaded}

\begin{verbatim}
##    C1 C2
## R1  1 26
## R2 24 68
\end{verbatim}

\begin{Shaded}
\begin{Highlighting}[]
\CommentTok{# Lists ---- }
\NormalTok{list1 <-}\StringTok{ }\KeywordTok{list}\NormalTok{(}\DataTypeTok{name =} \StringTok{"Maria"}\NormalTok{, }\DataTypeTok{mynumbers =}\NormalTok{ vector1, }\DataTypeTok{mymatrix =}\NormalTok{ matrix1, }\DataTypeTok{age =} \FloatTok{5.3}\NormalTok{); list1}
\end{Highlighting}
\end{Shaded}

\begin{verbatim}
## $name
## [1] "Maria"
## 
## $mynumbers
## [1]  1.0  2.0  5.3  6.0 -2.0  4.0
## 
## $mymatrix
##      [,1] [,2] [,3] [,4]
## [1,]    1    6   11   16
## [2,]    2    7   12   17
## [3,]    3    8   13   18
## [4,]    4    9   14   19
## [5,]    5   10   15   20
## 
## $age
## [1] 5.3
\end{verbatim}

\begin{Shaded}
\begin{Highlighting}[]
\NormalTok{list1[[}\DecValTok{2}\NormalTok{]]}
\end{Highlighting}
\end{Shaded}

\begin{verbatim}
## [1]  1.0  2.0  5.3  6.0 -2.0  4.0
\end{verbatim}

\begin{Shaded}
\begin{Highlighting}[]
\CommentTok{# Data Frames ----}
\NormalTok{d <-}\StringTok{ }\KeywordTok{c}\NormalTok{(}\DecValTok{1}\NormalTok{, }\DecValTok{2}\NormalTok{, }\DecValTok{3}\NormalTok{, }\DecValTok{4}\NormalTok{) }\CommentTok{# What type of vector?}
\NormalTok{e <-}\StringTok{ }\KeywordTok{c}\NormalTok{(}\StringTok{"red"}\NormalTok{, }\StringTok{"white"}\NormalTok{, }\StringTok{"red"}\NormalTok{, }\OtherTok{NA}\NormalTok{) }\CommentTok{# What type of vector?}
\NormalTok{f <-}\StringTok{ }\KeywordTok{c}\NormalTok{(}\OtherTok{TRUE}\NormalTok{, }\OtherTok{TRUE}\NormalTok{, }\OtherTok{TRUE}\NormalTok{, }\OtherTok{FALSE}\NormalTok{) }\CommentTok{# What type of vector?}
\NormalTok{dataframe1 <-}\StringTok{ }\KeywordTok{data.frame}\NormalTok{(d,e,f) }\CommentTok{#}
\NormalTok{dataframe1}
\end{Highlighting}
\end{Shaded}

\begin{verbatim}
##   d     e     f
## 1 1   red  TRUE
## 2 2 white  TRUE
## 3 3   red  TRUE
## 4 4  <NA> FALSE
\end{verbatim}

\begin{Shaded}
\begin{Highlighting}[]
\KeywordTok{names}\NormalTok{(dataframe1) <-}\StringTok{ }\KeywordTok{c}\NormalTok{(}\StringTok{"ID"}\NormalTok{,}\StringTok{"Color"}\NormalTok{,}\StringTok{"Passed"}\NormalTok{); }\KeywordTok{View}\NormalTok{(dataframe1) }\CommentTok{# }

\NormalTok{dataframe1[}\DecValTok{1}\OperatorTok{:}\DecValTok{2}\NormalTok{,] }\CommentTok{# }
\end{Highlighting}
\end{Shaded}

\begin{verbatim}
##   ID Color Passed
## 1  1   red   TRUE
## 2  2 white   TRUE
\end{verbatim}

\begin{Shaded}
\begin{Highlighting}[]
\NormalTok{dataframe1[}\KeywordTok{c}\NormalTok{(}\StringTok{"ID"}\NormalTok{,}\StringTok{"Passed"}\NormalTok{)] }\CommentTok{# }
\end{Highlighting}
\end{Shaded}

\begin{verbatim}
##   ID Passed
## 1  1   TRUE
## 2  2   TRUE
## 3  3   TRUE
## 4  4  FALSE
\end{verbatim}

\begin{Shaded}
\begin{Highlighting}[]
\NormalTok{dataframe1}\OperatorTok{$}\NormalTok{ID}
\end{Highlighting}
\end{Shaded}

\begin{verbatim}
## [1] 1 2 3 4
\end{verbatim}

Question: How do the different types of data appear in the Environment
tab?

\begin{quote}
Answer:
\end{quote}

Question: In the R chunk below, write ``dataframe1\$''. Press
\texttt{tab} after you type the dollar sign. What happens?

\begin{quote}
Answer:
\end{quote}

\hypertarget{coding-challenge}{%
\subsubsection{Coding challenge}\label{coding-challenge}}

Find a ten-day forecast of temperatures (Fahrenheit) for Durham, North
Carolina. Create two vectors, one representing the high temperature on
each of the ten days and one representing the low.

Now, create two additional vectors that include the ten-day forecast for
the high and low temperatures in Celsius.

Combine your four vectors into a data frame and add informative column
names.

Use the common functions \texttt{summary} and \texttt{sd} to obtain
basic data summaries of the ten-day forecast. How would you call these
functions differently for the entire data frame vs.~a single column?
Attempt to demonstrate both options below.

\hypertarget{packages}{%
\subsection{Packages}\label{packages}}

The Packages tab in the notebook stores the packages that you have saved
in your system. A checkmark next to each package indicates whether the
package has been loaded into your current R session. Given that R is an
open source software, users can create packages that have specific
functionalities, with complicated code ``packaged'' into a simple
commands.

If you want to use a specific package that is not in your libaray
already, you need to install it. You can do this in two ways:

\begin{enumerate}
\def\labelenumi{\arabic{enumi}.}
\item
  Click the install button in the packages tab. Type the package name,
  which should autocomplete below (case matters). Make sure to check
  ``intall dependencies,'' which will also install packages that your
  new package uses.
\item
  Type \texttt{install.packages("packagename")} into your R chunk or
  console. It will then appear in your packages list. You only need to
  do this once.
\end{enumerate}

If a package is already installed, you will need to load it every
session. You can do this in two ways:

\begin{enumerate}
\def\labelenumi{\arabic{enumi}.}
\item
  Click the box next to the package name in the Packages tab.
\item
  Type \texttt{library(packagename)} into your R chunk or console.
\end{enumerate}

\begin{Shaded}
\begin{Highlighting}[]
\CommentTok{# We will use the packages dplyr and ggplot2 regularly. }
\CommentTok{#install.packages("dplyr") }
\CommentTok{#install.packages("ggplot2")}
\CommentTok{# comment out install commands, use only when needed and re-comment}

\KeywordTok{library}\NormalTok{(dplyr)}
\end{Highlighting}
\end{Shaded}

\begin{verbatim}
## 
## Attaching package: 'dplyr'
\end{verbatim}

\begin{verbatim}
## The following objects are masked from 'package:stats':
## 
##     filter, lag
\end{verbatim}

\begin{verbatim}
## The following objects are masked from 'package:base':
## 
##     intersect, setdiff, setequal, union
\end{verbatim}

\begin{Shaded}
\begin{Highlighting}[]
\KeywordTok{library}\NormalTok{(ggplot2)}

\CommentTok{# Some packages are umbrellas under which other packages are loaded}
\CommentTok{#install.packages("tidyverse")}
\KeywordTok{library}\NormalTok{(tidyverse)}
\end{Highlighting}
\end{Shaded}

\begin{verbatim}
## -- Attaching packages ------------------------------------------------------------- tidyverse 1.3.0 --
\end{verbatim}

\begin{verbatim}
## v tibble  2.1.3     v purrr   0.3.3
## v tidyr   1.0.0     v stringr 1.4.0
## v readr   1.3.1     v forcats 0.4.0
\end{verbatim}

\begin{verbatim}
## -- Conflicts ---------------------------------------------------------------- tidyverse_conflicts() --
## x dplyr::filter() masks stats::filter()
## x dplyr::lag()    masks stats::lag()
\end{verbatim}

Question: What happens in the console when you load a package?

\begin{quote}
Answer:
\end{quote}

\hypertarget{tips-and-tricks}{%
\subsection{Tips and Tricks}\label{tips-and-tricks}}

\begin{itemize}
\item
  Sequential section headers can be created by using at least four -, =,
  and \# characters.
\item
  The command \texttt{require(packagename)} will also load a package,
  but it will not give any error or warning messages if there is an
  issue.
\item
  You may be asked to restart R when installing or updating packages.
  Feel free to say no, as this will obviously slow your progress.
  However, if the functionality of your new package isn't working
  properly, try restarting R as a first step.
\item
  If asked ``Do you want to install from sources the packages which
  needs compilation?'', type \texttt{yes} into the console.
\item
  You should only install packages once on your machine. If you store
  \texttt{install.packages} in your R chunks/scripts, comment these
  lines out.
\item
  Update your packages regularly!
\end{itemize}

\end{document}
