\PassOptionsToPackage{unicode=true}{hyperref} % options for packages loaded elsewhere
\PassOptionsToPackage{hyphens}{url}
%
\documentclass[]{article}
\usepackage{lmodern}
\usepackage{amssymb,amsmath}
\usepackage{ifxetex,ifluatex}
\usepackage{fixltx2e} % provides \textsubscript
\ifnum 0\ifxetex 1\fi\ifluatex 1\fi=0 % if pdftex
  \usepackage[T1]{fontenc}
  \usepackage[utf8]{inputenc}
  \usepackage{textcomp} % provides euro and other symbols
\else % if luatex or xelatex
  \usepackage{unicode-math}
  \defaultfontfeatures{Ligatures=TeX,Scale=MatchLowercase}
\fi
% use upquote if available, for straight quotes in verbatim environments
\IfFileExists{upquote.sty}{\usepackage{upquote}}{}
% use microtype if available
\IfFileExists{microtype.sty}{%
\usepackage[]{microtype}
\UseMicrotypeSet[protrusion]{basicmath} % disable protrusion for tt fonts
}{}
\IfFileExists{parskip.sty}{%
\usepackage{parskip}
}{% else
\setlength{\parindent}{0pt}
\setlength{\parskip}{6pt plus 2pt minus 1pt}
}
\usepackage{hyperref}
\hypersetup{
            pdftitle={4: Data Exploration},
            pdfauthor={Environmental Data Analytics \textbar{} Kateri Salk},
            pdfborder={0 0 0},
            breaklinks=true}
\urlstyle{same}  % don't use monospace font for urls
\usepackage[margin=2.54cm]{geometry}
\usepackage{color}
\usepackage{fancyvrb}
\newcommand{\VerbBar}{|}
\newcommand{\VERB}{\Verb[commandchars=\\\{\}]}
\DefineVerbatimEnvironment{Highlighting}{Verbatim}{commandchars=\\\{\}}
% Add ',fontsize=\small' for more characters per line
\usepackage{framed}
\definecolor{shadecolor}{RGB}{248,248,248}
\newenvironment{Shaded}{\begin{snugshade}}{\end{snugshade}}
\newcommand{\AlertTok}[1]{\textcolor[rgb]{0.94,0.16,0.16}{#1}}
\newcommand{\AnnotationTok}[1]{\textcolor[rgb]{0.56,0.35,0.01}{\textbf{\textit{#1}}}}
\newcommand{\AttributeTok}[1]{\textcolor[rgb]{0.77,0.63,0.00}{#1}}
\newcommand{\BaseNTok}[1]{\textcolor[rgb]{0.00,0.00,0.81}{#1}}
\newcommand{\BuiltInTok}[1]{#1}
\newcommand{\CharTok}[1]{\textcolor[rgb]{0.31,0.60,0.02}{#1}}
\newcommand{\CommentTok}[1]{\textcolor[rgb]{0.56,0.35,0.01}{\textit{#1}}}
\newcommand{\CommentVarTok}[1]{\textcolor[rgb]{0.56,0.35,0.01}{\textbf{\textit{#1}}}}
\newcommand{\ConstantTok}[1]{\textcolor[rgb]{0.00,0.00,0.00}{#1}}
\newcommand{\ControlFlowTok}[1]{\textcolor[rgb]{0.13,0.29,0.53}{\textbf{#1}}}
\newcommand{\DataTypeTok}[1]{\textcolor[rgb]{0.13,0.29,0.53}{#1}}
\newcommand{\DecValTok}[1]{\textcolor[rgb]{0.00,0.00,0.81}{#1}}
\newcommand{\DocumentationTok}[1]{\textcolor[rgb]{0.56,0.35,0.01}{\textbf{\textit{#1}}}}
\newcommand{\ErrorTok}[1]{\textcolor[rgb]{0.64,0.00,0.00}{\textbf{#1}}}
\newcommand{\ExtensionTok}[1]{#1}
\newcommand{\FloatTok}[1]{\textcolor[rgb]{0.00,0.00,0.81}{#1}}
\newcommand{\FunctionTok}[1]{\textcolor[rgb]{0.00,0.00,0.00}{#1}}
\newcommand{\ImportTok}[1]{#1}
\newcommand{\InformationTok}[1]{\textcolor[rgb]{0.56,0.35,0.01}{\textbf{\textit{#1}}}}
\newcommand{\KeywordTok}[1]{\textcolor[rgb]{0.13,0.29,0.53}{\textbf{#1}}}
\newcommand{\NormalTok}[1]{#1}
\newcommand{\OperatorTok}[1]{\textcolor[rgb]{0.81,0.36,0.00}{\textbf{#1}}}
\newcommand{\OtherTok}[1]{\textcolor[rgb]{0.56,0.35,0.01}{#1}}
\newcommand{\PreprocessorTok}[1]{\textcolor[rgb]{0.56,0.35,0.01}{\textit{#1}}}
\newcommand{\RegionMarkerTok}[1]{#1}
\newcommand{\SpecialCharTok}[1]{\textcolor[rgb]{0.00,0.00,0.00}{#1}}
\newcommand{\SpecialStringTok}[1]{\textcolor[rgb]{0.31,0.60,0.02}{#1}}
\newcommand{\StringTok}[1]{\textcolor[rgb]{0.31,0.60,0.02}{#1}}
\newcommand{\VariableTok}[1]{\textcolor[rgb]{0.00,0.00,0.00}{#1}}
\newcommand{\VerbatimStringTok}[1]{\textcolor[rgb]{0.31,0.60,0.02}{#1}}
\newcommand{\WarningTok}[1]{\textcolor[rgb]{0.56,0.35,0.01}{\textbf{\textit{#1}}}}
\usepackage{graphicx,grffile}
\makeatletter
\def\maxwidth{\ifdim\Gin@nat@width>\linewidth\linewidth\else\Gin@nat@width\fi}
\def\maxheight{\ifdim\Gin@nat@height>\textheight\textheight\else\Gin@nat@height\fi}
\makeatother
% Scale images if necessary, so that they will not overflow the page
% margins by default, and it is still possible to overwrite the defaults
% using explicit options in \includegraphics[width, height, ...]{}
\setkeys{Gin}{width=\maxwidth,height=\maxheight,keepaspectratio}
\setlength{\emergencystretch}{3em}  % prevent overfull lines
\providecommand{\tightlist}{%
  \setlength{\itemsep}{0pt}\setlength{\parskip}{0pt}}
\setcounter{secnumdepth}{0}
% Redefines (sub)paragraphs to behave more like sections
\ifx\paragraph\undefined\else
\let\oldparagraph\paragraph
\renewcommand{\paragraph}[1]{\oldparagraph{#1}\mbox{}}
\fi
\ifx\subparagraph\undefined\else
\let\oldsubparagraph\subparagraph
\renewcommand{\subparagraph}[1]{\oldsubparagraph{#1}\mbox{}}
\fi

% set default figure placement to htbp
\makeatletter
\def\fps@figure{htbp}
\makeatother


\title{4: Data Exploration}
\author{Environmental Data Analytics \textbar{} Kateri Salk}
\date{Spring 2020}

\begin{document}
\maketitle

\hypertarget{lesson-objectives}{%
\subsection{Lesson Objectives}\label{lesson-objectives}}

\begin{enumerate}
\def\labelenumi{\arabic{enumi}.}
\tightlist
\item
  Set up a data analysis session in RStudio
\item
  Import and explore datasets in R
\item
  Apply data exploration skills to a real-world example dataset
\end{enumerate}

\hypertarget{best-practices-in-r}{%
\subsection{Best Practices in R}\label{best-practices-in-r}}

In many situations in data analytics, you may be expected to work from
multiple computers or share projects among multiple users. A few general
best practices will avoid common pitfalls related to collaborative work.

\hypertarget{set-your-working-directory}{%
\subsubsection{Set your working
directory}\label{set-your-working-directory}}

A session in RStudio will always function by mapping to a specific
folder in your computer, called the \emph{working directory}. All
navigation between folders and files will happen relative to this
working directory. When you open an R project, your working directory
will automatically set to the folder that holds the project file. If you
open an R script or RMarkdown document directly by double-clicking the
file, your working directory will automatically set to the folder that
holds that file. It is a good idea to note with a comment at the top of
your file which working directory you intend the user to designate.

In this course, we will always open the R project file for the course,
and additional navigation of the working directory will happen from that
folder. To check your working directory, use the following R command:

\begin{Shaded}
\begin{Highlighting}[]
\CommentTok{# Working directory should be set to the parent folder for the Environmental Data Analytics Course, i.e., the folder that houses the R Project file. Running getwd() will let you know which file path you ar working from}

\KeywordTok{getwd}\NormalTok{()}
\end{Highlighting}
\end{Shaded}

\begin{verbatim}
## [1] "/Users/amandabraun/Documents/Classes Spring 2020/Data Analytics/Environmental_Data_Analytics_2020"
\end{verbatim}

If your working directory is not set to the folder you want, you have
several options. The first is to directly code your working directory.
You may do this by defining an absolute file path (below). What are the
pitfalls of using an absolute file path?

\begin{Shaded}
\begin{Highlighting}[]
\CommentTok{# Absolute file path is commented out}
\CommentTok{#setwd("/Users/katerisalk/Documents/Duke/Courses/Environmental_Data_Analytics")}
\end{Highlighting}
\end{Shaded}

You may change your working directory without coding by going to the
Session menu in RStudio and navigating to the Set Working Directory tab.
From there, you may select from a series of options to reset your
working directory.

Another option is to use the R package \texttt{here}. We will not be
using this option in class, but it is growing quite popular among R
users. A more detailed description and rationale can be found here:
\url{https://github.com/jennybc/here_here}.

\hypertarget{load-your-packages}{%
\subsubsection{Load your packages}\label{load-your-packages}}

At the top of your R scripts, you should load any packages that need to
be used for that R script. A common issue that arises is that packages
will be loaded in the middle of the code, making it difficult to run
specific chunks of code without scrolling to make sure all necessary
packages are loaded. For example, the tidyverse package is one that we
will use regularly in class.

At the same time, you should also load your theme if you are doing any
data visualization with ggplot. More on this later.

\begin{Shaded}
\begin{Highlighting}[]
\CommentTok{# Load package, don't print this on the pdf }
\KeywordTok{library}\NormalTok{(tidyverse)}
\end{Highlighting}
\end{Shaded}

\hypertarget{import-your-datasets}{%
\subsubsection{Import your datasets}\label{import-your-datasets}}

Datasets can be imported into R. Good data practices dictate that raw
data (from yourself or others) should not be changed and re-saved within
the spreadsheet, but rather the data should be changed with reproducible
techniques and saved as a new file. Note: data should be saved in
nonproprietary formats, namely .csv or .txt files rather than .xls or
.xlsx files.

To read in a data file, you may specify a file path with an
\emph{absolute} or a \emph{relative} file path. As above with your
working directory, it is a better practice to use a relative directory.
To navigate a relative file path, use \texttt{./} followed by the tab
key to navigate forward in the folder structure, and use \texttt{../}
followed by the tab key to navigate back out of the folder structure.
For example, this lesson is located in the ``Lessons'' folder, and we
need to navigate into the ``Data'' folder. After clicking the correct
folder, use \texttt{/} and press tab again to continue the process.

You may also import datasets from the Files tab, but this is not
recommended since this is not reproducible.

\begin{Shaded}
\begin{Highlighting}[]
\CommentTok{# Absolute file path (not recommended)}
\CommentTok{#read.csv("/Users/katerisalk/Documents/Duke/Courses/Environmental_Data_Analytics/Data/Raw/USGS_Site02085000_Flow_Raw.csv")}

\CommentTok{# Relative file path (friendly for users regardless of machine) preferred method }
\NormalTok{USGS.flow.data <-}\StringTok{ }\KeywordTok{read.csv}\NormalTok{(}\StringTok{"./Data/Raw/USGS_Site02085000_Flow_Raw.csv"}\NormalTok{)}

\CommentTok{# What happens if we don't assign a name to our imported dataset?}
\CommentTok{#read.csv("../Data/Raw/USGS_Site02085000_Flow_Raw.csv")}

\CommentTok{# Another option is to choose with your browser}
\CommentTok{#read.csv(file.choose())}

\CommentTok{# To import .txt files, use read.table rather than read.csv}
\CommentTok{#read.table()}
\end{Highlighting}
\end{Shaded}

\hypertarget{explore-your-dataset}{%
\subsection{EXPLORE YOUR DATASET}\label{explore-your-dataset}}

Take a moment to read through the README file associated with the USGS
dataset on discharge at the Eno River. Where can you find this file? How
does the placement and information found in this file relate to the best
practices for reproducible data analysis? \textgreater{} ANSWER: The
file is in the Data folder in a parallel file with metadata. This is
best practice because it is easy to find within a parallel source. The
data itself contains little useful info, column names are not useful
without context.

\begin{Shaded}
\begin{Highlighting}[]
\KeywordTok{View}\NormalTok{(USGS.flow.data)}
\CommentTok{# Alternate option: click on data frame in Environment tab}

\KeywordTok{class}\NormalTok{(USGS.flow.data)}
\end{Highlighting}
\end{Shaded}

\begin{verbatim}
## [1] "data.frame"
\end{verbatim}

\begin{Shaded}
\begin{Highlighting}[]
\KeywordTok{colnames}\NormalTok{(USGS.flow.data)}
\end{Highlighting}
\end{Shaded}

\begin{verbatim}
##  [1] "agency_cd"              "site_no"                "datetime"              
##  [4] "X165986_00060_00001"    "X165986_00060_00001_cd" "X165987_00060_00002"   
##  [7] "X165987_00060_00002_cd" "X84936_00060_00003"     "X84936_00060_00003_cd" 
## [10] "X84937_00065_00001"     "X84937_00065_00001_cd"  "X84938_00065_00002"    
## [13] "X84938_00065_00002_cd"  "X84939_00065_00003"     "X84939_00065_00003_cd"
\end{verbatim}

\begin{Shaded}
\begin{Highlighting}[]
\CommentTok{# Rename columns}
\KeywordTok{colnames}\NormalTok{(USGS.flow.data) <-}\StringTok{ }\KeywordTok{c}\NormalTok{(}\StringTok{"agency_cd"}\NormalTok{, }\StringTok{"site_no"}\NormalTok{, }\StringTok{"datetime"}\NormalTok{, }
                              \StringTok{"discharge.max"}\NormalTok{, }\StringTok{"discharge.max.approval"}\NormalTok{, }
                              \StringTok{"discharge.min"}\NormalTok{, }\StringTok{"discharge.min.approval"}\NormalTok{, }
                              \StringTok{"discharge.mean"}\NormalTok{, }\StringTok{"discharge.mean.approval"}\NormalTok{, }
                              \StringTok{"gage.height.max"}\NormalTok{, }\StringTok{"gage.height.max.approval"}\NormalTok{, }
                              \StringTok{"gage.height.min"}\NormalTok{, }\StringTok{"gage.height.min.approval"}\NormalTok{, }
                              \StringTok{"gage.height.mean"}\NormalTok{, }\StringTok{"gage.height.mean.approval"}\NormalTok{)}
\KeywordTok{str}\NormalTok{(USGS.flow.data) }\CommentTok{#numberic returns NA values if there is blank, character or factor blank}
\end{Highlighting}
\end{Shaded}

\begin{verbatim}
## 'data.frame':    33690 obs. of  15 variables:
##  $ agency_cd                : Factor w/ 1 level "USGS": 1 1 1 1 1 1 1 1 1 1 ...
##  $ site_no                  : int  2085000 2085000 2085000 2085000 2085000 2085000 2085000 2085000 2085000 2085000 ...
##  $ datetime                 : Factor w/ 33690 levels "1/1/00","1/1/01",..: 2873 3896 4919 5198 5291 5384 5477 5570 5663 2966 ...
##  $ discharge.max            : num  NA NA NA NA NA NA NA NA NA NA ...
##  $ discharge.max.approval   : Factor w/ 3 levels "","A","P": 1 1 1 1 1 1 1 1 1 1 ...
##  $ discharge.min            : num  NA NA NA NA NA NA NA NA NA NA ...
##  $ discharge.min.approval   : Factor w/ 3 levels "","A","P": 1 1 1 1 1 1 1 1 1 1 ...
##  $ discharge.mean           : num  39 39 39 39 39 39 39 39 39 39 ...
##  $ discharge.mean.approval  : Factor w/ 4 levels "","A","A:e","P": 2 2 2 2 2 2 2 2 2 2 ...
##  $ gage.height.max          : num  NA NA NA NA NA NA NA NA NA NA ...
##  $ gage.height.max.approval : Factor w/ 3 levels "","A","P": 1 1 1 1 1 1 1 1 1 1 ...
##  $ gage.height.min          : num  NA NA NA NA NA NA NA NA NA NA ...
##  $ gage.height.min.approval : Factor w/ 3 levels "","A","P": 1 1 1 1 1 1 1 1 1 1 ...
##  $ gage.height.mean         : num  NA NA NA NA NA NA NA NA NA NA ...
##  $ gage.height.mean.approval: Factor w/ 3 levels "","A","P": 1 1 1 1 1 1 1 1 1 1 ...
\end{verbatim}

\begin{Shaded}
\begin{Highlighting}[]
\KeywordTok{dim}\NormalTok{(USGS.flow.data) }\CommentTok{#missed this - dimensions  }
\end{Highlighting}
\end{Shaded}

\begin{verbatim}
## [1] 33690    15
\end{verbatim}

\begin{Shaded}
\begin{Highlighting}[]
\KeywordTok{length}\NormalTok{(USGS.flow.data) }\CommentTok{#number of columns }
\end{Highlighting}
\end{Shaded}

\begin{verbatim}
## [1] 15
\end{verbatim}

\begin{Shaded}
\begin{Highlighting}[]
\KeywordTok{nrow}\NormalTok{(USGS.flow.data) }\CommentTok{#number of rows }
\end{Highlighting}
\end{Shaded}

\begin{verbatim}
## [1] 33690
\end{verbatim}

\begin{Shaded}
\begin{Highlighting}[]
\KeywordTok{head}\NormalTok{(USGS.flow.data) }\CommentTok{#default is first 6 rows of the data}
\end{Highlighting}
\end{Shaded}

\begin{verbatim}
##   agency_cd site_no datetime discharge.max discharge.max.approval discharge.min
## 1      USGS 2085000  10/1/27            NA                                   NA
## 2      USGS 2085000  10/2/27            NA                                   NA
## 3      USGS 2085000  10/3/27            NA                                   NA
## 4      USGS 2085000  10/4/27            NA                                   NA
## 5      USGS 2085000  10/5/27            NA                                   NA
## 6      USGS 2085000  10/6/27            NA                                   NA
##   discharge.min.approval discharge.mean discharge.mean.approval gage.height.max
## 1                                    39                       A              NA
## 2                                    39                       A              NA
## 3                                    39                       A              NA
## 4                                    39                       A              NA
## 5                                    39                       A              NA
## 6                                    39                       A              NA
##   gage.height.max.approval gage.height.min gage.height.min.approval
## 1                                       NA                         
## 2                                       NA                         
## 3                                       NA                         
## 4                                       NA                         
## 5                                       NA                         
## 6                                       NA                         
##   gage.height.mean gage.height.mean.approval
## 1               NA                          
## 2               NA                          
## 3               NA                          
## 4               NA                          
## 5               NA                          
## 6               NA
\end{verbatim}

\begin{Shaded}
\begin{Highlighting}[]
\KeywordTok{head}\NormalTok{(USGS.flow.data, }\DecValTok{10}\NormalTok{) }\CommentTok{# pulls the first 10 rows of the data }
\end{Highlighting}
\end{Shaded}

\begin{verbatim}
##    agency_cd site_no datetime discharge.max discharge.max.approval
## 1       USGS 2085000  10/1/27            NA                       
## 2       USGS 2085000  10/2/27            NA                       
## 3       USGS 2085000  10/3/27            NA                       
## 4       USGS 2085000  10/4/27            NA                       
## 5       USGS 2085000  10/5/27            NA                       
## 6       USGS 2085000  10/6/27            NA                       
## 7       USGS 2085000  10/7/27            NA                       
## 8       USGS 2085000  10/8/27            NA                       
## 9       USGS 2085000  10/9/27            NA                       
## 10      USGS 2085000 10/10/27            NA                       
##    discharge.min discharge.min.approval discharge.mean discharge.mean.approval
## 1             NA                                    39                       A
## 2             NA                                    39                       A
## 3             NA                                    39                       A
## 4             NA                                    39                       A
## 5             NA                                    39                       A
## 6             NA                                    39                       A
## 7             NA                                    39                       A
## 8             NA                                    39                       A
## 9             NA                                    39                       A
## 10            NA                                    39                       A
##    gage.height.max gage.height.max.approval gage.height.min
## 1               NA                                       NA
## 2               NA                                       NA
## 3               NA                                       NA
## 4               NA                                       NA
## 5               NA                                       NA
## 6               NA                                       NA
## 7               NA                                       NA
## 8               NA                                       NA
## 9               NA                                       NA
## 10              NA                                       NA
##    gage.height.min.approval gage.height.mean gage.height.mean.approval
## 1                                         NA                          
## 2                                         NA                          
## 3                                         NA                          
## 4                                         NA                          
## 5                                         NA                          
## 6                                         NA                          
## 7                                         NA                          
## 8                                         NA                          
## 9                                         NA                          
## 10                                        NA
\end{verbatim}

\begin{Shaded}
\begin{Highlighting}[]
\KeywordTok{tail}\NormalTok{(USGS.flow.data, }\DecValTok{5}\NormalTok{) }\CommentTok{# pulls the last five rows of the data }
\end{Highlighting}
\end{Shaded}

\begin{verbatim}
##       agency_cd site_no datetime discharge.max discharge.max.approval
## 33686      USGS 2085000 12/22/19            NA                       
## 33687      USGS 2085000 12/23/19            NA                       
## 33688      USGS 2085000 12/24/19            NA                       
## 33689      USGS 2085000 12/25/19            NA                       
## 33690      USGS 2085000 12/26/19            NA                       
##       discharge.min discharge.min.approval discharge.mean
## 33686            NA                                  18.1
## 33687            NA                                  18.6
## 33688            NA                                  18.8
## 33689            NA                                  16.6
## 33690            NA                                  15.1
##       discharge.mean.approval gage.height.max gage.height.max.approval
## 33686                       P              NA                         
## 33687                       P              NA                         
## 33688                       P              NA                         
## 33689                       P              NA                         
## 33690                       P              NA                         
##       gage.height.min gage.height.min.approval gage.height.mean
## 33686              NA                                      1.93
## 33687              NA                                      1.94
## 33688              NA                                      1.95
## 33689              NA                                      1.91
## 33690              NA                                      1.88
##       gage.height.mean.approval
## 33686                         P
## 33687                         P
## 33688                         P
## 33689                         P
## 33690                         P
\end{verbatim}

\begin{Shaded}
\begin{Highlighting}[]
\NormalTok{USGS.flow.data[}\DecValTok{30000}\OperatorTok{:}\DecValTok{30005}\NormalTok{, }\KeywordTok{c}\NormalTok{(}\DecValTok{3}\NormalTok{, }\DecValTok{8}\NormalTok{, }\DecValTok{14}\NormalTok{)] }\CommentTok{#rows 30000 to 30005, missed the rest }
\end{Highlighting}
\end{Shaded}

\begin{verbatim}
##       datetime discharge.mean gage.height.mean
## 30000 11/18/09           27.5             1.72
## 30001 11/19/09           31.6             1.80
## 30002 11/20/09           37.1             1.88
## 30003 11/21/09           32.1             1.80
## 30004 11/22/09           23.7             1.66
## 30005 11/23/09          337.0             3.87
\end{verbatim}

\begin{Shaded}
\begin{Highlighting}[]
\KeywordTok{class}\NormalTok{(USGS.flow.data}\OperatorTok{$}\NormalTok{datetime) }\CommentTok{#why do you copy and paste this as opposed to running it? how do you know to do this?}
\end{Highlighting}
\end{Shaded}

\begin{verbatim}
## [1] "factor"
\end{verbatim}

\begin{Shaded}
\begin{Highlighting}[]
\KeywordTok{class}\NormalTok{(USGS.flow.data}\OperatorTok{$}\NormalTok{discharge.mean)}
\end{Highlighting}
\end{Shaded}

\begin{verbatim}
## [1] "numeric"
\end{verbatim}

\begin{Shaded}
\begin{Highlighting}[]
\KeywordTok{class}\NormalTok{(USGS.flow.data}\OperatorTok{$}\NormalTok{gage.height.mean)}
\end{Highlighting}
\end{Shaded}

\begin{verbatim}
## [1] "numeric"
\end{verbatim}

\begin{Shaded}
\begin{Highlighting}[]
\KeywordTok{summary}\NormalTok{(USGS.flow.data) }\CommentTok{#}
\end{Highlighting}
\end{Shaded}

\begin{verbatim}
##  agency_cd       site_no           datetime     discharge.max    
##  USGS:33690   Min.   :2085000   1/1/00 :    1   Min.   :   0.26  
##               1st Qu.:2085000   1/1/01 :    1   1st Qu.:   7.23  
##               Median :2085000   1/1/02 :    1   Median :  21.15  
##               Mean   :2085000   1/1/03 :    1   Mean   :  88.15  
##               3rd Qu.:2085000   1/1/04 :    1   3rd Qu.:  59.80  
##               Max.   :2085000   1/1/05 :    1   Max.   :4730.00  
##                                 (Other):33684   NA's   :28342    
##  discharge.max.approval discharge.min     discharge.min.approval
##   :28342                Min.   :   0.09    :28342               
##  A: 5347                1st Qu.:   4.38   A: 5347               
##  P:    1                Median :  12.60   P:    1               
##                         Mean   :  30.46                         
##                         3rd Qu.:  34.80                         
##                         Max.   :1460.00                         
##                         NA's   :28342                           
##  discharge.mean    discharge.mean.approval gage.height.max 
##  Min.   :   0.02      : 5108               Min.   : 0.890  
##  1st Qu.:   9.30   A  :28265               1st Qu.: 1.490  
##  Median :  24.00   A:e:  294               Median : 1.830  
##  Mean   :  59.48   P  :   23               Mean   : 2.124  
##  3rd Qu.:  54.00                           3rd Qu.: 2.310  
##  Max.   :4600.00                           Max.   :17.020  
##  NA's   :5108                              NA's   :28229   
##  gage.height.max.approval gage.height.min gage.height.min.approval
##   :28229                  Min.   :0.840    :28229                 
##  A: 5460                  1st Qu.:1.380   A: 5460                 
##  P:    1                  Median :1.650   P:    1                 
##                           Mean   :1.736                           
##                           3rd Qu.:2.030                           
##                           Max.   :9.190                           
##                           NA's   :28229                           
##  gage.height.mean gage.height.mean.approval
##  Min.   : 0.870    :24870                  
##  1st Qu.: 1.450   A: 8797                  
##  Median : 1.770   P:   23                  
##  Mean   : 1.952                            
##  3rd Qu.: 2.200                            
##  Max.   :15.040                            
##  NA's   :24870
\end{verbatim}

What happened to blank cells in the spreadsheet when they were imported
into R? \textgreater{} Answer: if they were numeric, they appear as Na,
if they are factor or character they come back as 1

\hypertarget{adjusting-datasets}{%
\subsection{Adjusting Datasets}\label{adjusting-datasets}}

\hypertarget{removing-nas}{%
\subsubsection{Removing NAs}\label{removing-nas}}

Notice in our dataset that our discharge and gage height observations
have many NAs, meaning no measurement was recorded for a specific day.
In some cases, it might be in our best interest to remove NAs from a
dataset. Removing NAs or not will depend on your research question.

\begin{Shaded}
\begin{Highlighting}[]
\KeywordTok{summary}\NormalTok{(USGS.flow.data}\OperatorTok{$}\NormalTok{discharge.mean)}
\end{Highlighting}
\end{Shaded}

\begin{verbatim}
##    Min. 1st Qu.  Median    Mean 3rd Qu.    Max.    NA's 
##    0.02    9.30   24.00   59.48   54.00 4600.00    5108
\end{verbatim}

\begin{Shaded}
\begin{Highlighting}[]
\KeywordTok{summary}\NormalTok{(USGS.flow.data}\OperatorTok{$}\NormalTok{gage.height.mean)}
\end{Highlighting}
\end{Shaded}

\begin{verbatim}
##    Min. 1st Qu.  Median    Mean 3rd Qu.    Max.    NA's 
##   0.870   1.450   1.770   1.952   2.200  15.040   24870
\end{verbatim}

Question: What types of research questions might make it favorable to
remove NAs from a dataset, and what types of research questions might
make it favorable to retain NAs in the dataset? Sometimes knowing NA
exists is informative, presence/absence data. Sometimes can only run
whole set of data, if you remove the NAs.

\begin{quote}
Answer:
\end{quote}

\begin{Shaded}
\begin{Highlighting}[]
\NormalTok{USGS.flow.data.complete <-}\StringTok{ }\KeywordTok{na.omit}\NormalTok{(USGS.flow.data)}
\KeywordTok{dim}\NormalTok{(USGS.flow.data) }\CommentTok{#unomitted NA}
\end{Highlighting}
\end{Shaded}

\begin{verbatim}
## [1] 33690    15
\end{verbatim}

\begin{Shaded}
\begin{Highlighting}[]
\KeywordTok{dim}\NormalTok{(USGS.flow.data.complete) }\CommentTok{#ommitted NAs}
\end{Highlighting}
\end{Shaded}

\begin{verbatim}
## [1] 5342   15
\end{verbatim}

\begin{Shaded}
\begin{Highlighting}[]
\KeywordTok{mean}\NormalTok{(USGS.flow.data.complete}\OperatorTok{$}\NormalTok{discharge.mean)}
\end{Highlighting}
\end{Shaded}

\begin{verbatim}
## [1] 51.08613
\end{verbatim}

\begin{Shaded}
\begin{Highlighting}[]
\KeywordTok{sd}\NormalTok{(USGS.flow.data.complete}\OperatorTok{$}\NormalTok{discharge.mean)}
\end{Highlighting}
\end{Shaded}

\begin{verbatim}
## [1] 137.2094
\end{verbatim}

\begin{Shaded}
\begin{Highlighting}[]
\KeywordTok{summary}\NormalTok{(USGS.flow.data.complete}\OperatorTok{$}\NormalTok{discharge.mean)}
\end{Highlighting}
\end{Shaded}

\begin{verbatim}
##     Min.  1st Qu.   Median     Mean  3rd Qu.     Max. 
##    0.220    5.683   16.600   51.086   44.800 3270.000
\end{verbatim}

\hypertarget{formatting-dates}{%
\subsubsection{Formatting dates}\label{formatting-dates}}

R will often import dates as factors or characters rather than dates. To
fix, this we need to tell R that it is looking at dates. We also need to
specify the format the dates are in. By default, if you don't provide a
format, R will attempt to use \%Y-\%m-\%d or \%Y/\%m/\%d as a default.
Note: if you are working collaboratively in an international setting,
using a year-month-day format in spreadsheets is the least ambiguous of
date formats. Make sure to check whether month-day-year or
day-month-year is used in an ambiguously formatted spreadsheet.

Formatting of dates in R:

\%d day as number (0-31) \%m month (00-12, can be e.g., 01 or 1) \%y
2-digit year \%Y 4-digit year \%a abbreviated weekday \%A unabbreviated
weekday \%b abbreviated month \%B unabbreviated month

In some cases when dates are provided as integers, you may need to
provide an origin for your dates. Beware: the ``origin'' date for Excel
(Windows), Excel (Mac), R, and MATLAB all have different origin dates.
Google this if it comes up.

Original date is January 1st, 1970.

\begin{Shaded}
\begin{Highlighting}[]
\KeywordTok{help}\NormalTok{(as.Date)}

\CommentTok{# Adjust date formatting for today}
\CommentTok{# Write code for three different date formats. }
\CommentTok{# An example is provided to get you started.}
\CommentTok{# (code must be uncommented)}
\NormalTok{today <-}\StringTok{ }\KeywordTok{Sys.Date}\NormalTok{()}
\CommentTok{#format(today, format = "%B")}
\CommentTok{#format(today, format = "%Y")}
\CommentTok{#format(today, format = "%a")}
\CommentTok{#format(today, format = "%b")}

\NormalTok{USGS.flow.data}\OperatorTok{$}\NormalTok{datetime <-}\StringTok{ }\KeywordTok{as.Date}\NormalTok{(USGS.flow.data}\OperatorTok{$}\NormalTok{datetime, }\DataTypeTok{format =} \StringTok{"%m/%d/%y"}\NormalTok{) }
\end{Highlighting}
\end{Shaded}

Note that for every date prior to 1969, R has assigned the date in the
2000s rather than the 1900s. This can be fixed with an \texttt{ifelse}
statement inside a function. Run through the code below and write what
is happening in the comment above each line.

\begin{Shaded}
\begin{Highlighting}[]
\CommentTok{#whaterformat the data is in, not what R has }

\NormalTok{USGS.flow.data}\OperatorTok{$}\NormalTok{datetime <-}\StringTok{ }\KeywordTok{format}\NormalTok{(USGS.flow.data}\OperatorTok{$}\NormalTok{datetime, }\DataTypeTok{format =} \StringTok{"%m/%d/%y"}\NormalTok{)}

\CommentTok{#}
\NormalTok{create.early.dates <-}\StringTok{ }\NormalTok{(}\ControlFlowTok{function}\NormalTok{(d) \{}
       \KeywordTok{paste0}\NormalTok{(}\KeywordTok{ifelse}\NormalTok{(d }\OperatorTok{>}\StringTok{ }\DecValTok{181231}\NormalTok{,}\StringTok{"19"}\NormalTok{,}\StringTok{"20"}\NormalTok{),d)}
\NormalTok{       \})}
\CommentTok{#}
\NormalTok{USGS.flow.data}\OperatorTok{$}\NormalTok{datetime <-}\StringTok{ }\KeywordTok{create.early.dates}\NormalTok{(USGS.flow.data}\OperatorTok{$}\NormalTok{datetime)}

\CommentTok{#}
\NormalTok{USGS.flow.data}\OperatorTok{$}\NormalTok{datetime <-}\StringTok{ }\KeywordTok{as.Date}\NormalTok{(USGS.flow.data}\OperatorTok{$}\NormalTok{datetime, }\DataTypeTok{format =} \StringTok{"%Y%m%d"}\NormalTok{) }
\end{Highlighting}
\end{Shaded}

\hypertarget{saving-datasets}{%
\subsection{Saving datasets}\label{saving-datasets}}

We just edited our raw dataset into a processed form. We may want to
return to this processed dataset later, which will be easier to do if we
save it as a spreadsheet.

\begin{Shaded}
\begin{Highlighting}[]
\KeywordTok{write.csv}\NormalTok{(USGS.flow.data, }\DataTypeTok{file =} \StringTok{"./Data/Processed/USGS_Site02085000_Flow_Processed.csv"}\NormalTok{, }\DataTypeTok{row.names=}\OtherTok{FALSE}\NormalTok{)}
\end{Highlighting}
\end{Shaded}

\hypertarget{tips-and-tricks}{%
\subsection{Tips and Tricks}\label{tips-and-tricks}}

\hypertarget{knitting}{%
\subsubsection{Knitting}\label{knitting}}

\begin{itemize}
\tightlist
\item
  In the Knit menu in the Editor, you will need to specify whether your
  knit directory should be the document directory or the project
  directory. If your document is not knitting correctly, try switching
  between the document directory and project directory as a first
  troubleshooting option.
\end{itemize}

\hypertarget{spreadsheets}{%
\subsubsection{Spreadsheets}\label{spreadsheets}}

*Files should be saved as .csv or .txt for easy import into R. Note that
complex formatting, including formulas in Excel, are not saved when
spreadsheets are converted to comma separated or text formats (i.e.,
values alone are saved).

*The first row is reserved for column headers.

*A secondary row for column headers (e.g., units) should not be used if
data are being imported into R. Incorporate units into the first row
column headers if necessary.

*Short names are preferred for column headers, to the extent they are
informative. Additional information can be stored in comments within R
scripts and/or in README files.

*Spaces in column names will be replaced with a \texttt{.} when imported
into R. When designing spreadsheets, avoid spaces in column headers.

*Avoid symbols in column headers. This can cause issues when importing
into R.

\end{document}
