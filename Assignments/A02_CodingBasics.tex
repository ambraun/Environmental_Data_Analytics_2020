\PassOptionsToPackage{unicode=true}{hyperref} % options for packages loaded elsewhere
\PassOptionsToPackage{hyphens}{url}
%
\documentclass[]{article}
\usepackage{lmodern}
\usepackage{amssymb,amsmath}
\usepackage{ifxetex,ifluatex}
\usepackage{fixltx2e} % provides \textsubscript
\ifnum 0\ifxetex 1\fi\ifluatex 1\fi=0 % if pdftex
  \usepackage[T1]{fontenc}
  \usepackage[utf8]{inputenc}
  \usepackage{textcomp} % provides euro and other symbols
\else % if luatex or xelatex
  \usepackage{unicode-math}
  \defaultfontfeatures{Ligatures=TeX,Scale=MatchLowercase}
\fi
% use upquote if available, for straight quotes in verbatim environments
\IfFileExists{upquote.sty}{\usepackage{upquote}}{}
% use microtype if available
\IfFileExists{microtype.sty}{%
\usepackage[]{microtype}
\UseMicrotypeSet[protrusion]{basicmath} % disable protrusion for tt fonts
}{}
\IfFileExists{parskip.sty}{%
\usepackage{parskip}
}{% else
\setlength{\parindent}{0pt}
\setlength{\parskip}{6pt plus 2pt minus 1pt}
}
\usepackage{hyperref}
\hypersetup{
            pdftitle={Assignment 2: Coding Basics},
            pdfauthor={Amanda Braun},
            pdfborder={0 0 0},
            breaklinks=true}
\urlstyle{same}  % don't use monospace font for urls
\usepackage[margin=2.54cm]{geometry}
\usepackage{color}
\usepackage{fancyvrb}
\newcommand{\VerbBar}{|}
\newcommand{\VERB}{\Verb[commandchars=\\\{\}]}
\DefineVerbatimEnvironment{Highlighting}{Verbatim}{commandchars=\\\{\}}
% Add ',fontsize=\small' for more characters per line
\usepackage{framed}
\definecolor{shadecolor}{RGB}{248,248,248}
\newenvironment{Shaded}{\begin{snugshade}}{\end{snugshade}}
\newcommand{\AlertTok}[1]{\textcolor[rgb]{0.94,0.16,0.16}{#1}}
\newcommand{\AnnotationTok}[1]{\textcolor[rgb]{0.56,0.35,0.01}{\textbf{\textit{#1}}}}
\newcommand{\AttributeTok}[1]{\textcolor[rgb]{0.77,0.63,0.00}{#1}}
\newcommand{\BaseNTok}[1]{\textcolor[rgb]{0.00,0.00,0.81}{#1}}
\newcommand{\BuiltInTok}[1]{#1}
\newcommand{\CharTok}[1]{\textcolor[rgb]{0.31,0.60,0.02}{#1}}
\newcommand{\CommentTok}[1]{\textcolor[rgb]{0.56,0.35,0.01}{\textit{#1}}}
\newcommand{\CommentVarTok}[1]{\textcolor[rgb]{0.56,0.35,0.01}{\textbf{\textit{#1}}}}
\newcommand{\ConstantTok}[1]{\textcolor[rgb]{0.00,0.00,0.00}{#1}}
\newcommand{\ControlFlowTok}[1]{\textcolor[rgb]{0.13,0.29,0.53}{\textbf{#1}}}
\newcommand{\DataTypeTok}[1]{\textcolor[rgb]{0.13,0.29,0.53}{#1}}
\newcommand{\DecValTok}[1]{\textcolor[rgb]{0.00,0.00,0.81}{#1}}
\newcommand{\DocumentationTok}[1]{\textcolor[rgb]{0.56,0.35,0.01}{\textbf{\textit{#1}}}}
\newcommand{\ErrorTok}[1]{\textcolor[rgb]{0.64,0.00,0.00}{\textbf{#1}}}
\newcommand{\ExtensionTok}[1]{#1}
\newcommand{\FloatTok}[1]{\textcolor[rgb]{0.00,0.00,0.81}{#1}}
\newcommand{\FunctionTok}[1]{\textcolor[rgb]{0.00,0.00,0.00}{#1}}
\newcommand{\ImportTok}[1]{#1}
\newcommand{\InformationTok}[1]{\textcolor[rgb]{0.56,0.35,0.01}{\textbf{\textit{#1}}}}
\newcommand{\KeywordTok}[1]{\textcolor[rgb]{0.13,0.29,0.53}{\textbf{#1}}}
\newcommand{\NormalTok}[1]{#1}
\newcommand{\OperatorTok}[1]{\textcolor[rgb]{0.81,0.36,0.00}{\textbf{#1}}}
\newcommand{\OtherTok}[1]{\textcolor[rgb]{0.56,0.35,0.01}{#1}}
\newcommand{\PreprocessorTok}[1]{\textcolor[rgb]{0.56,0.35,0.01}{\textit{#1}}}
\newcommand{\RegionMarkerTok}[1]{#1}
\newcommand{\SpecialCharTok}[1]{\textcolor[rgb]{0.00,0.00,0.00}{#1}}
\newcommand{\SpecialStringTok}[1]{\textcolor[rgb]{0.31,0.60,0.02}{#1}}
\newcommand{\StringTok}[1]{\textcolor[rgb]{0.31,0.60,0.02}{#1}}
\newcommand{\VariableTok}[1]{\textcolor[rgb]{0.00,0.00,0.00}{#1}}
\newcommand{\VerbatimStringTok}[1]{\textcolor[rgb]{0.31,0.60,0.02}{#1}}
\newcommand{\WarningTok}[1]{\textcolor[rgb]{0.56,0.35,0.01}{\textbf{\textit{#1}}}}
\usepackage{graphicx,grffile}
\makeatletter
\def\maxwidth{\ifdim\Gin@nat@width>\linewidth\linewidth\else\Gin@nat@width\fi}
\def\maxheight{\ifdim\Gin@nat@height>\textheight\textheight\else\Gin@nat@height\fi}
\makeatother
% Scale images if necessary, so that they will not overflow the page
% margins by default, and it is still possible to overwrite the defaults
% using explicit options in \includegraphics[width, height, ...]{}
\setkeys{Gin}{width=\maxwidth,height=\maxheight,keepaspectratio}
\setlength{\emergencystretch}{3em}  % prevent overfull lines
\providecommand{\tightlist}{%
  \setlength{\itemsep}{0pt}\setlength{\parskip}{0pt}}
\setcounter{secnumdepth}{0}
% Redefines (sub)paragraphs to behave more like sections
\ifx\paragraph\undefined\else
\let\oldparagraph\paragraph
\renewcommand{\paragraph}[1]{\oldparagraph{#1}\mbox{}}
\fi
\ifx\subparagraph\undefined\else
\let\oldsubparagraph\subparagraph
\renewcommand{\subparagraph}[1]{\oldsubparagraph{#1}\mbox{}}
\fi

% set default figure placement to htbp
\makeatletter
\def\fps@figure{htbp}
\makeatother


\title{Assignment 2: Coding Basics}
\author{Amanda Braun}
\date{}

\begin{document}
\maketitle

\hypertarget{overview}{%
\subsection{OVERVIEW}\label{overview}}

This exercise accompanies the lessons in Environmental Data Analytics on
coding basics.

\hypertarget{directions}{%
\subsection{Directions}\label{directions}}

\begin{enumerate}
\def\labelenumi{\arabic{enumi}.}
\tightlist
\item
  Change ``Student Name'' on line 3 (above) with your name.
\item
  Work through the steps, \textbf{creating code and output} that fulfill
  each instruction.
\item
  Be sure to \textbf{answer the questions} in this assignment document.
\item
  When you have completed the assignment, \textbf{Knit} the text and
  code into a single PDF file.
\item
  After Knitting, submit the completed exercise (PDF file) to the
  dropbox in Sakai. Add your last name into the file name (e.g.,
  ``Salk\_A02\_CodingBasics.Rmd'') prior to submission.
\end{enumerate}

The completed exercise is due on Tuesday, January 21 at 1:00 pm.

\hypertarget{basics-day-1}{%
\subsection{Basics Day 1}\label{basics-day-1}}

\begin{enumerate}
\def\labelenumi{\arabic{enumi}.}
\item
  Generate a sequence of numbers from one to 100, increasing by fours.
  Assign this sequence a name.
\item
  Compute the mean and median of this sequence.
\item
  Ask R to determine whether the mean is greater than the median.
\item
  Insert comments in your code to describe what you are doing.
\end{enumerate}

\begin{Shaded}
\begin{Highlighting}[]
\CommentTok{#1. seq is function to generate a sequence, (from, to, by)  why doens't it produce the numbers? }
\NormalTok{amanda_sequence <-}\StringTok{ }\KeywordTok{seq}\NormalTok{(}\DecValTok{1}\NormalTok{, }\DecValTok{100}\NormalTok{, }\DecValTok{4}\NormalTok{) }

\CommentTok{#2. }
\KeywordTok{mean}\NormalTok{(amanda_sequence)}
\end{Highlighting}
\end{Shaded}

\begin{verbatim}
## [1] 49
\end{verbatim}

\begin{Shaded}
\begin{Highlighting}[]
\KeywordTok{median}\NormalTok{(amanda_sequence) }
\end{Highlighting}
\end{Shaded}

\begin{verbatim}
## [1] 49
\end{verbatim}

\begin{Shaded}
\begin{Highlighting}[]
\CommentTok{#3. }
\KeywordTok{mean}\NormalTok{(amanda_sequence) }\OperatorTok{<}\StringTok{ }\KeywordTok{median}\NormalTok{(amanda_sequence)}
\end{Highlighting}
\end{Shaded}

\begin{verbatim}
## [1] FALSE
\end{verbatim}

\hypertarget{basics-day-2}{%
\subsection{Basics Day 2}\label{basics-day-2}}

\begin{enumerate}
\def\labelenumi{\arabic{enumi}.}
\setcounter{enumi}{4}
\item
  Create a series of vectors, each with four components, consisting of
  (a) names of students, (b) test scores out of a total 100 points, and
  (c) whether or not they have passed the test (TRUE or FALSE) with a
  passing grade of 50.
\item
  Label each vector with a comment on what type of vector it is.
\item
  Combine each of the vectors into a data frame. Assign the data frame
  an informative name.
\item
  Label the columns of your data frame with informative titles.
\end{enumerate}

\begin{Shaded}
\begin{Highlighting}[]
\NormalTok{names <-}\StringTok{ }\KeywordTok{c}\NormalTok{(}\StringTok{"Amanda"}\NormalTok{, }\StringTok{"Rani"}\NormalTok{, }\StringTok{"Cate"}\NormalTok{, }\StringTok{"Rachel"}\NormalTok{) }
\NormalTok{names}
\end{Highlighting}
\end{Shaded}

\begin{verbatim}
## [1] "Amanda" "Rani"   "Cate"   "Rachel"
\end{verbatim}

\begin{Shaded}
\begin{Highlighting}[]
\CommentTok{#character vector }
\NormalTok{testscore <-}\StringTok{ }\KeywordTok{c}\NormalTok{(}\DecValTok{95}\NormalTok{, }\DecValTok{48}\NormalTok{, }\DecValTok{92}\NormalTok{, }\DecValTok{72}\NormalTok{)}
\NormalTok{testscore}
\end{Highlighting}
\end{Shaded}

\begin{verbatim}
## [1] 95 48 92 72
\end{verbatim}

\begin{Shaded}
\begin{Highlighting}[]
\CommentTok{#numeric vector }
\NormalTok{testpass <-}\StringTok{ }\KeywordTok{c}\NormalTok{(}\OtherTok{TRUE}\NormalTok{, }\OtherTok{FALSE}\NormalTok{, }\OtherTok{TRUE}\NormalTok{, }\OtherTok{TRUE}\NormalTok{)}
\NormalTok{testpass}
\end{Highlighting}
\end{Shaded}

\begin{verbatim}
## [1]  TRUE FALSE  TRUE  TRUE
\end{verbatim}

\begin{Shaded}
\begin{Highlighting}[]
\CommentTok{#logical vector }

\NormalTok{classscoresNovember10 <-}\StringTok{ }\KeywordTok{data.frame}\NormalTok{(names, testscore, testpass)}
\NormalTok{classscoresNovember10}
\end{Highlighting}
\end{Shaded}

\begin{verbatim}
##    names testscore testpass
## 1 Amanda        95     TRUE
## 2   Rani        48    FALSE
## 3   Cate        92     TRUE
## 4 Rachel        72     TRUE
\end{verbatim}

\begin{Shaded}
\begin{Highlighting}[]
\KeywordTok{names}\NormalTok{(classscoresNovember10) <-}\StringTok{ }\KeywordTok{c}\NormalTok{(}\StringTok{"Name"}\NormalTok{, }\StringTok{"Score"}\NormalTok{, }\StringTok{"Pass"}\NormalTok{); }\KeywordTok{View}\NormalTok{(classscoresNovember10)}
\end{Highlighting}
\end{Shaded}

\begin{enumerate}
\def\labelenumi{\arabic{enumi}.}
\setcounter{enumi}{8}
\tightlist
\item
  QUESTION: How is this data frame different from a matrix?
\end{enumerate}

\begin{quote}
Answer: Both data frames and matrixes represent two dimensional data
sets. A data frame can display multiple types of data - character,
logical, numeric. A matrix represents one type of data in a set number
of rows and columns.
\end{quote}

\begin{enumerate}
\def\labelenumi{\arabic{enumi}.}
\setcounter{enumi}{9}
\item
  Create a function with an if/else statement. Your function should
  determine whether a test score is a passing grade of 50 or above (TRUE
  or FALSE). You will need to choose either the \texttt{if} and
  \texttt{else} statements or the \texttt{ifelse} statement. Hint: Use
  \texttt{print}, not \texttt{return}. The name of your function should
  be informative.
\item
  Apply your function to the vector with test scores that you created in
  number 5.
\end{enumerate}

\begin{Shaded}
\begin{Highlighting}[]
\NormalTok{passingscore <-}\StringTok{ }\ControlFlowTok{function}\NormalTok{(x)\{}
  \ControlFlowTok{if}\NormalTok{ (x }\OperatorTok{<}\StringTok{ }\DecValTok{50}\NormalTok{) (}\DataTypeTok{x=}\OtherTok{FALSE}\NormalTok{) }\ControlFlowTok{else}\NormalTok{ (}\DataTypeTok{x=}\OtherTok{TRUE}\NormalTok{)\} }\CommentTok{# if and else function without print }
\KeywordTok{passingscore}\NormalTok{ (}\DecValTok{45}\NormalTok{) }
\end{Highlighting}
\end{Shaded}

\begin{verbatim}
## [1] FALSE
\end{verbatim}

\begin{Shaded}
\begin{Highlighting}[]
\KeywordTok{passingscore}\NormalTok{(testscore)}
\end{Highlighting}
\end{Shaded}

\begin{verbatim}
## Warning in if (x < 50) (x = FALSE) else (x = TRUE): the condition has length > 1
## and only the first element will be used
\end{verbatim}

\begin{verbatim}
## [1] TRUE
\end{verbatim}

\begin{Shaded}
\begin{Highlighting}[]
\NormalTok{passingscoreprint<-}\StringTok{ }\ControlFlowTok{function}\NormalTok{ (x) \{}\ControlFlowTok{if}\NormalTok{ (x}\OperatorTok{<}\DecValTok{50}\NormalTok{) }\KeywordTok{print}\NormalTok{(}\StringTok{"True"}\NormalTok{) }\ControlFlowTok{else} \KeywordTok{print}\NormalTok{(}\StringTok{"False"}\NormalTok{)\} }\CommentTok{# "if" and  "else" function with print }
\KeywordTok{passingscoreprint}\NormalTok{(}\DecValTok{45}\NormalTok{) }\CommentTok{#testing single object }
\end{Highlighting}
\end{Shaded}

\begin{verbatim}
## [1] "True"
\end{verbatim}

\begin{Shaded}
\begin{Highlighting}[]
\KeywordTok{passingscoreprint}\NormalTok{(testscore) }\CommentTok{#testing testscore vector }
\end{Highlighting}
\end{Shaded}

\begin{verbatim}
## Warning in if (x < 50) print("True") else print("False"): the condition has
## length > 1 and only the first element will be used
\end{verbatim}

\begin{verbatim}
## [1] "False"
\end{verbatim}

\begin{Shaded}
\begin{Highlighting}[]
\NormalTok{passing_score2print <-}\StringTok{ }\ControlFlowTok{function}\NormalTok{(x) \{}
  \KeywordTok{ifelse}\NormalTok{((x}\OperatorTok{<}\DecValTok{50}\NormalTok{), }
         \KeywordTok{print}\NormalTok{(}\StringTok{"False"}\NormalTok{), }\KeywordTok{print}\NormalTok{(}\StringTok{"True"}\NormalTok{))\} }\CommentTok{#ifelse function with print }
\KeywordTok{passing_score2print}\NormalTok{(}\DecValTok{6}\NormalTok{)}
\end{Highlighting}
\end{Shaded}

\begin{verbatim}
## [1] "False"
\end{verbatim}

\begin{verbatim}
## [1] "False"
\end{verbatim}

\begin{Shaded}
\begin{Highlighting}[]
\KeywordTok{passing_score2print}\NormalTok{(testscore)}
\end{Highlighting}
\end{Shaded}

\begin{verbatim}
## [1] "False"
## [1] "True"
\end{verbatim}

\begin{verbatim}
## [1] "True"  "False" "True"  "True"
\end{verbatim}

\begin{enumerate}
\def\labelenumi{\arabic{enumi}.}
\setcounter{enumi}{11}
\tightlist
\item
  QUESTION: Which option of \texttt{if} and \texttt{else} vs.
  \texttt{ifelse} worked? Why?
\end{enumerate}

\begin{quote}
Answer: \#``if'' and ``else'' work for one value, but do not work for a
vector. When ``if'' and ``else'' is used for a vector, an warning
message is produced that the conditions has a length \textgreater{} 1
and only the first element will be used, meaning that we cannot input a
vector with multiple objects into the function. \#``ifelse'' does work
for a vector both a single object and for a vector
\end{quote}

\end{document}
